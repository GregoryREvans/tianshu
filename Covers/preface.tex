\documentclass[10pt]{article}
\usepackage{fontspec}
\setmainfont[Ligatures=TeX]{Didot}
\usepackage[utf8]{inputenc}
\usepackage[papersize={11in, 17in}]{geometry}
\usepackage[absolute]{textpos}
\TPGrid[0.5in, 0.25in]{23}{24}
\usepackage{palatino}
\parindent=0pt
\parskip=12pt
\usepackage{nopageno}
\usepackage{graphicx}
\graphicspath{ {./images/} }
\usepackage{lilyglyphs}
\usepackage{amsmath}
\begin{document}

\vspace*{0.5\baselineskip}

\begin{center}
\huge FOREWORD
\end{center}

\begin{center}
Ti\=ansh\=u is the name of an art installation in the form of a book by artist Xu Bing filled with meaningless glyphs in the style of traditional Chinese characters. The term ti\=an sh\=u, which can be translated to mean "divine writing," originally referred to religious texts but is now used to mean "gibberish." A possible alternative title could be "Nonsense Writing."\\
\phantom{text} \hfill (G.R.E.)
  \end{center}
  
\vspace*{10\baselineskip}

\begin{center}
\huge PERFORMANCE NOTES
\end{center}

\begin{center}
\pmb{Microtones}:
\end{center}

\begin{center}
\includegraphics[width=0.5\textwidth]{microtones.png}
\end{center}

\vspace*{30\baselineskip}

\begin{center}
c.8'
\end{center}

\vspace*{2\baselineskip}

\begin{center}
$Tianshu$ is dedicated in admiration and friendship to Trevor Ba\v{c}a, Josiah Wolf Oberholtzer, and Jeffrey Trevi\~{n}o from whom I have learned so much.
\end{center}

\end{document}