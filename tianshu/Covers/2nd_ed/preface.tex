\documentclass[10pt]{article}
\usepackage{fontspec}
\usepackage[utf8]{inputenc}
\setmainfont{Didot}
%\usepackage[papersize={11in, 17in}]{geometry}
\usepackage[paperwidth=11in,paperheight=17in,margin=1in,headheight=0.0in,footskip=0.5in,includehead,includefoot,portrait]{geometry}
\usepackage[absolute]{textpos}
\TPGrid[0.5in, 0.25in]{23}{24}
\parindent=0pt
\parskip=12pt
\usepackage{nopageno}
\usepackage{graphicx}
\graphicspath{ {./images/} }
\usepackage{amsmath}
\usepackage{tikz}
\newcommand*\circled[1]{\tikz[baseline=(char.base)]{
            \node[shape=circle,draw,inner sep=1pt] (char) {#1};}}

\begin{document}

\begin{center}
\huge FOREWORD
\end{center}

\begin{center}
\leftskip1.5in
Ti\=ansh\=u is the name of an art installation in the form of a book by artist Xu Bing filled with meaningless glyphs in the style of traditional Chinese characters, referred to in English as ``A Book from the Sky.'' The term ti\=an sh\=u, which can be translated to mean ``divine writing," originally referred to religious texts but is now used to mean ``gibberish." A possible alternative title could be ``Nonsense Writing." The first title of this installation, and the Chinese subtitle of this piece, can be translated to ``Mirror to Analyze the World: The Century's Final Volume.'' I have elected to typeset the Chinese title and subtitle of this piece in traditional Chinese characters rather than simplified characters in the spirit of the traditional nature of Xu Bing's woodcut printing used in the making of Ti\=ansh\=u, although it is possible that this is not the correct choice.
\rightskip\leftskip
\phantom{text} \hfill (G.R.E.)
  \end{center}
  
\vspace*{1\baselineskip}

\begin{center}
\huge PERFORMANCE NOTES
\end{center}

\begin{center}
Score is transposed.
\end{center}

\begin{center}
\pmb{Microtones}:
\end{center}

\begin{center}
\includegraphics[width=0.5\textwidth]{microtones.png}
\end{center}

\begin{center}
Accidentals apply only to the pitch which they immediately precede.
\end{center}

\vspace*{0.3\baselineskip}

\begin{center}
\huge INSTRUMENTATION
\end{center}
\begin{center}
Flute \\
Clarinet in Bb \\
Bassoon \\
Horn in F \\
Trumpet in Bb \\
Trombone \\
Tuba \\
2 Violins \\
Viola \\
Violoncello \\
Contrabass
\end{center}

\vspace*{10\baselineskip}

\begin{center}
c.8'
\end{center}

\vspace*{10\baselineskip}

\begin{center}
\leftskip2.3in
$Tianshu$ is dedicated in admiration and friendship to Trevor Ba\v{c}a, Josiah Wolf Oberholtzer, and Jeffrey Trevi\~{n}o from whom I have learned so much.
\rightskip\leftskip
\end{center}

\end{document}